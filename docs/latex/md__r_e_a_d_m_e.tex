The goal of this project is to develop an open-\/source lattice K\+MC simulation software package for modeling organic semiconductor materials and devices, such as O\+P\+Vs, O\+L\+E\+Ds, and more. The software is being developed in C++11 and optimized for efficient execution on high performance computing clusters using M\+PI. This software package uses object-\/oriented design and extends the \href{https://github.com/MikeHeiber/KMC_Lattice}{\tt K\+M\+C\+\_\+\+Lattice} framework.

\subsection*{Current Status\+:}

The current version (Excimontec v0.\+4-\/alpha) is built with K\+M\+C\+\_\+\+Lattice v2.\+0-\/alpha.\+5 and allows the user to perform several simulation tests relevant for O\+PV and O\+L\+ED devices. N\+O\+TE\+: This software is still in the alpha phase of development, and as such, there may still be bugs that need to be squashed. Please report any bugs or submit feature requests in the Issues section.

\paragraph*{Implemented Major Features\+:}


\begin{DoxyItemize}
\item Adjustable periodic boundary conditions in all three directions allow users to perform 1D, 2D, or 3D simulations.
\item Choose between several film architectures, including a neat film, bilayer film, or random blend film.
\item Import bulk heterojunction morphologies generated by \href{https://github.com/MikeHeiber/Ising_OPV}{\tt Ising\+\_\+\+O\+PV v3.\+2 and v4}.
\item Donor and acceptor materials can take on an uncorrelated Gaussian D\+OS, a correlated Gaussian D\+OS with different correlation functions, or an uncorrelated exponential D\+OS model.
\item Dynamics test simulations can be performed to generate exciton and charge carrier density transients that can be used to model exciton dissociation and charge carrier recombination kinetics.
\item Time-\/of-\/flight charge transport simulations of electrons or holes can be performed on neat, random blend, or bulk heterojunction blend films.
\item \hyperlink{class_exciton}{Exciton} diffusion simulations can be performed on any film architecture.
\item Internal quantum efficiency simulations can be performed on bilayer, random blend, or bulk heterojunction blend films.
\item Simulate complex exciton dynamics with events for intersystem crossing between singlet and triplet states as well as exciton-\/exciton and exciton-\/polaron annihilation events.
\item Choose between Miller-\/\+Abrahams or Marcus models for polaron hopping.
\item Charge carrier delocalization can be modeled with a spherical Gaussian delocalization model.
\end{DoxyItemize}

\subsection*{How to try Excimontec?}

\paragraph*{Compiling}

Compiling requires an M\+PI library for parallel processing.

More information about common packages can be found here\+:
\begin{DoxyItemize}
\item \href{http://www.open-mpi.org/}{\tt http\+://www.\+open-\/mpi.\+org/}
\item \href{http://www.mpich.org/}{\tt http\+://www.\+mpich.\+org/}
\item \href{http://mvapich.cse.ohio-state.edu/}{\tt http\+://mvapich.\+cse.\+ohio-\/state.\+edu/}
\end{DoxyItemize}

\paragraph*{Usage}

Excimontec.\+exe takes one required input argument, which is the filename of the input parameter file.

An example parameter file is provided with parameters\+\_\+default.\+txt

As an example, to create a single simulation instance on a single processor, the command is\+: \begin{quote}
Excimontec.\+exe parameters\+\_\+default.\+txt \end{quote}


To run in a parallel processing environment and create 10 simulations on 10 processors to gather more statistics, an example run command is\+: \begin{quote}
mpiexec -\/n 10 Excimontec.\+exe parameters\+\_\+default.\+txt \end{quote}


M\+PI execution commands can be implemented into batch scripts for running Excimontec in a supercomputing environment.

\paragraph*{Output}

Excimontec will create a number of different output files depending which test is chosen\+:
\begin{DoxyItemize}
\item results\#.txt -- This text file will contain the results for each processor where the \# will be replaced by the processor ID.
\item analysis\+\_\+summary.\+txt -- When M\+PI is enabled, this text file will contain average final results from all of the processors.
\item dynamics\+\_\+average\+\_\+transients.\+txt -- When performing a dynamics test, calculated exciton, electron, and hole transients will be output to this file.
\item To\+F\+\_\+average\+\_\+transients.\+txt -- When performing a time-\/of-\/flight charge transport test, calculated current transients, mobility relaxation transients, and energy relaxation transients will be output to this file.
\item To\+F\+\_\+transit\+\_\+time\+\_\+dist.\+txt -- When performing a time-\/of-\/flight charge transport test, the resulting polaron transit time probability distribution will be output to this file.
\item To\+F\+\_\+results.\+txt -- When performing a time-\/of-\/flight charge transport test, the resulting quantitative results are put into this parsable delimited results file.
\item Charge\+\_\+extraction\+\_\+map\#.txt -- When performing a time-\/of-\/flight or I\+QE test, the x-\/y locations where charges are extracted from the lattice are saving into this map file. 
\end{DoxyItemize}